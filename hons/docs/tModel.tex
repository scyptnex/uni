\chapter{Mathematical Model}
\label{chapModel}

This chapter details the mathematical background for various processes used in the SDFSimulator.

\section{Checkpoint Recomputation}
The checkpoint recomputation (CR) method keeps a checkpoint for the last successful execution's state.
When a steady-state schedule is completed correctly the checkpoint is updated.
If it fails the system can return to the last successful state and continue.

As stated in Chapter \ref{chapSystems} several assumptions are made as to the nature of CR fault tolerance.
We understand the expected steady-state execution time of a CR fault tolerant system as follows:
\begin{itemize}
	\item The makespan (including the overhead of making checkpoints) of a single successful execution is known statically, we shall call this $m$.
	\item For the duration of any makespan, the probability of any single processor or communication failure is known statically.
			Hence the probability that at least one component fails can be worked out, we shall call this $p$.
	\item The time required to reset the system to the last checkpointed state is known statically, we shall call this $r$.
\end{itemize}

Since we are only concerned if a single component fails, we can take the generous presumption of failure independence (only the first failure matters).
This makes calculating the total expected execution time $t$ simple.
\begin{align}
	\nonumber t_i & = (1-p)m + p(m + r + t_{i+1}) \\
	\nonumber t & = t_0 \\
	\nonumber \mbox{Which can be re-written as:}& \\
	\nonumber t & = \left({(m+r)\displaystyle\sum\limits_{i=0}^\infty p^i }\right) - r \\
	\nonumber & = (m+r)\left({{1}\over{1-p}}\right) - r\\
	\nonumber & = {{m+rp}\over{1-p}}
\end{align}

\section{Task Replication}

The replication method initiates multiple versions of the same computations and runs them in parallel.
The schedule is executed as though there were only one version of each actor, however extra communication and computation is taken up by each actor: first to distribute output tokens to all versions of their successor nodes, and second to analyse each input token and choose only those that come from non-faulty predecessors.
For our implementation, the system only totally fails if all duplicates of one actor reside on faulting processors.

We understand the failure probability of a replication FT system as follows:
\begin{itemize}
	\item We are distributing $a$ actors, each duplicated $d$ times, hence $n = ad$ nodes in total, onto $m$ processors or machines.
	\item The probability of any processor failing during one steady-state execution is $p$, for simplicity we assume failures are independent and uniformly likely.
\end{itemize}

This leads to some useful observations about the costs of this FT method:
\begin{itemize}
	\item Communication between nodes increases with more duplicates.
			Every version of each actor communicates with every version of its predecessor, hence the communication increases by $d^2$.
	\item The system's throughput decreases with more duplicates if $m < n$.
			This leads to assigning multiple actors to the same processor, again it is a trivial observation.
\end{itemize}

We are interested with the probability of the system failing completely.
Each computer has a small chance of failing per steady-state execution $p$, then the probability that the computer is still running after i iterations is $P = p_i = (1-p)^i$.
In the case where each processor only executes one node, the system will fail if all duplicates of any one actor fail after i iterations.
All duplicates of a single actor fail with probability $(1-P)^d$.
One or more actors fail if not no actors fail.
A single actor does not fail with probability $1-(1-P)^d$ as above, hence at least one actor fails with probability $1-(1-(1-P)^d)^a$.
In summary, the system fails before $i$ iterations with probability:
\begin{align}
	\nonumber 1-(1-(1-(1-p)^i)^d)^a.
\end{align}

\section{AMPL Model}

An integer programming formulation of the Robust Actor Assignment Problem (RAAP, formalised in Chapter \ref{chapHardness}) is one of the contributions of this thesis.
This section describes the model, its meaning and formulation.

\subsection{Description}

The AMPL model is used to compute a cost optimal assignment of actors to processors whilst ensuring that duplicate actors are not assigned to the same processor.
Though designed for the replication FT mechanism, this model can also be used to assign actors optimally in the recomputation FT mechanism by removing the duplicate constraints (see Section \ref{secModVars}).

An ILP solver (in our case, glpsol) is used in conjunction with the model to generate optimal solutions to the RAAP.
For simplicity {\em optimal} in relation to the global sum of costs, rather than the more complicated, though useful method, of counting only the cost of the most expensive parallel unit.
Possible future work may include reformulating this problem to minimise parallel makespan, that is, by distributing actors in such a way that no single processor performs a disproportionately large number of computations.
The optimal solution can be compared with assignments derived from other heuristics for experimental purposes, however due to the intractable nature of the problem only small instances can be compared in this way.

In general terms we are trying to place computing actors onto the various processors such that each actor is placed on a processor which can execute it quickly, and the actors do not need to send much data over the network.
When an actor is placed on a processor, it incurs some cost to execute, these costs may vary between actors and processors without constraint.
Every pair of actors also incurs some cost depending on how much they communicate with each other and which processors they are placed on.
Intuitively we might say that two actors assigned to the same processor do not incur this communication cost, however we may wish to make actors pay different amounts for reserving space in a machine's memory buffer, so once again no limitations are placed on these communication costs.

\subsection{Problem variables}
\label{secModVars}


The parameters and variables for this model are explained below.
Some parameters make use of the \invoke{a}{p} and \commu{a}{b}{p}{q} functions, these are described in Section \ref{secSystemNPM}.
To simplify the notation of matrices we use the notation $\mathbf{A}_{r,c}$ to denote the value in row $r$ and column $c$ of the $\mathbf{A}$ matrix.

\begin{itemize}
	\item There are $n$ duplicates of all actors, that is, if there were $a$ actors being duplicated $d$ times each then $n = ad$.
	\item There are $p$ processors.
	\item The set of all duplicates of all actors is represented by $N = [0, 1, ..., n-1]$.
	\item The set of all processors is represented by $P = [0, 1, ..., p-1]$
	\item The cost of invoking actor $a$ on processor $p$, is stored in the matrix $\mathbf{I}$, a $n \times p$ matrix where $\mathbf{I}_{a,p} = \invoke{a}{p}$.
	\item The cost of communication between actors $a$ and $b$ on processors $p$ and $q$ respectively is the matrix $\mathbf{C}$, a $n \times n \times p \times p$ tensor where $\mathbf{C}_{a,b,p,q} = \commu{a}{b}{p}{q}$.
	\item Which actors are duplicates of which is recorded in the matrix $\mathbf{D}$, a $n \times n$ matrix where $\mathbf{D}_{i,j} = \mbox{$\left\{ 
		\begin{array}{l l}
			1 \quad & \mbox{if $i$ is a duplicate of $j$}\\
			0 & \mbox{otherwise}\\ \end{array} \right.$}$ \\ Note $\forall i, j \in N : D_{i,j} = D_{j,i}$ 
	\item The assignment of actors is determined by the solver in the matrix $\mathbf{X}$, a $n \times p$ matrix where $\mathbf{X}_{a,p} = \mbox{$\left\{ 
		\begin{array}{l l}
			1 \quad & \mbox{if $a$ will be run on $p$}\\
			0 & \mbox{otherwise}\\ \end{array} \right.$}$
\end{itemize}

\subsection{ILP Formulation}

In its simplest form the RAAP is modelled as:

\begin{align}
	\nonumber \min & \sum_{a \in N; p \in P} \mathbf{I}_{a,p}\mathbf{X}_{a,p} + \sum_{a,b \in N; p,q \in P} \mathbf{C}_{a,b,p,q}\mathbf{X}_{a,p}\mathbf{X}_{b,q} \\
	\nonumber s.t. &  \\
	\nonumber & \forall a \in N : \sum_{p \in P}\mathbf{X}_{a,p} = 1 \\
	\nonumber & \forall a,b \in N : \sum_{p \in P}\mathbf{X}_{a,p}\mathbf{X}_{b,p}\mathbf{D}_{a,b} = 0
\end{align}

The objective function minimises the cost of invoking all actors on their assigned processors and communicating between all actors on their processors.
The constraints are, first all actors must be assigned to exactly one processor, second all actors which are duplicates of each other cannot be assigned to the same processor (this lower sum will be zero unless $a$ and $b$ are both assigned to $p$ and they are duplicates).

\subsection{Constraint Linearisation}
\label{secModLin}

Viewing the model above we immediately see the problem of quadratic terms.
The minimisation of the objective function is determined by the minimisation of two dependant assignments (for the communication cost part of the sum).
In order to solve this problem we must linearise this quadratic term.

We use the standard method for linearising this quantity.
Since the $\mathbf{X}$ matrix can only take values $\{0, 1\}$, the multiplication of these numbers can be represented by linear constraints.
We let $\mathbf{Y}$ be the $n \times n \times p \times p$ tensor, where $\mathbf{Y}_{a,b,p,q} = \mathbf{X}_{a,p} \times \mathbf{X}_{b,q}$.
The equivalent expression in linear form is:
\begin{align}
	\nonumber \forall a,b \in N : \forall p,q \in P : & \mathbf{Y}_{a,b,p,q} \leq \mathbf{X}_{a,p} \\
	\nonumber \forall a,b \in N : \forall p,q \in P : & \mathbf{Y}_{a,b,p,q} \leq \mathbf{X}_{b,q} \\
	\nonumber \forall a,b \in N : \forall p,q \in P : & \mathbf{X}_{a,p} + \mathbf{X}_{b,q} - 1 \leq \mathbf{Y}_{a,b,p,q}
\end{align}

The equivalence of this linearised form to the quadratic term can be shown trivially by enumerating all the possible combinations of the $\mathbf{X}$ parts, which can only be zero and one.

Hence the modified ILP formulation without quadratic terms is:

\begin{align}
	\nonumber \min & \sum_{a \in N; p \in P} \mathbf{I}_{a,p}\mathbf{X}_{a,p} + \sum_{a,b \in N; p,q \in P} \mathbf{C}_{a,b,p,q}\mathbf{Y}_{a,b,p,q} \\
	\nonumber s.t. &  \\
	\nonumber & \forall a \in N : \sum_{p \in P}\mathbf{X}_{a,p} = 1 \\
	\nonumber & \forall a,b \in N : \forall p,q \in P : \quad \mathbf{Y}_{a,b,p,q} \leq \mathbf{X}_{a,p} \\
	\nonumber & \forall a,b \in N : \forall p,q \in P : \quad \mathbf{Y}_{a,b,p,q} \leq \mathbf{X}_{b,q} \\
	\nonumber & \forall a,b \in N : \forall p,q \in P : \quad \mathbf{X}_{a,p} + \mathbf{X}_{b,q} - 1 \leq \mathbf{Y}_{a,b,p,q} \\
	\nonumber & \forall a,b \in N : \sum_{p \in P}\mathbf{Y}_{a,b,p,p}\mathbf{D}_{a,b} = 0
\end{align}

The AMPL-encoded version of this ILP can be found in Appendix \ref{adxModel}
