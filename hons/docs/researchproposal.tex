\documentstyle{article}

\author{Nic Hollingum - 308193415}
\title{Research Methods - Assignment 3}

\addtolength{\oddsidemargin}{-.875in}
\addtolength{\evensidemargin}{-.875in}
\addtolength{\textwidth}{1.75in}
\addtolength{\topmargin}{-.875in}
\addtolength{\textheight}{1.75in}

\begin{document}

\maketitle

\section*{Research Proposal}

\subsection*{Contribution}
This project intends to determine some of the limitations and costs of implementing fault tolerant protocols for Synchronous Dataflow (SDF) programs in cloud and grid environments.
Current implementations of SDF systems \cite{mal08, thies02, thies10} do not explicitly guarantee fault tolerance in execution.
This is an acceptable shortfall in the case of standalone systems, where a 'node failure' means the whole computer dies, and so no mechanisms can exist to recover from such a failure, the computation must be rerun.
However in the cloud/grid context mean time to failure (MTTF) can be as short as 100h \cite{ree06}.
As such, mechanisms must be put in place to ensure successful computation given a few node failures.

Large parallel programming tools such as Google's MapReduce \cite{dea08} have fault-tolerance inbuilt.
In the case of MapReduce this means task-reallocation, though other fault tolerant systems may have software redundancies \cite{ran75}, checkpointing \cite{li10} or process duplication \cite{lit07}.

This research will look at 2 generalizations of fault-tolerance mechanisms, re-computation and task-duplication.
We examine the effects of ensuring higher levels of fault tolerance (that is, reducing the probability of failure) on the bandwidth and time requirements of systems.
We formulate this tradeoff as a 3-way optimization problem, and use known heuristics to solve the integer-programming formulations of the scheduling problem.
Finally we test the predictions of the model experimentally using a simulator, which represents cloud and grid platforms as a generic ``n-process'' machine.

This research aims to formally specify the relationship between fault-tolerance and the makespan/bandwidth of a SDF program.


\subsection*{Methodology}

Scheduling heiristics have been used to solve the assignment problem with respect to makespan \cite{len87} or communication \cite{boy01} (or, rarely, both) for some time.
We first formulate the assignment problem which achieves all 3 constraints, makespan, communication, and fault toelrance.
Once we have this formulation we examine the approximation bounds for their distance from optimality.

We also compare the conditions under which either one of the fault-tolerance mechanisms may lead to more optimal solutions.
It is known that both mechanisms necessarily increase the makespan of the problem, and the communication bandwidth.
However it is unclear which mechanism produces larger overheads, or if ther is some meeting point at which a previously less costly fault-tolerance mechanism becomes more optimal.
We model the increases in communication bandwidth and makespan for each of the methods to gain an idea of how each affects the result.

\subsection*{Evaluation}

The theoretical data above is corroborated against experimental data using an sdfSimulator.
Though it is called ``simulator'' the framework is able to perform arbitrary computations in a SDF model, the n-process machine is simulated and makespan/communication costs are determined by examining the simulated network.
This simulator will also allow for the injection of faults, both intentionally and at random, to test the resiliance of each of the fault-tolerant mechanisms.
The simulator is used to run real SDF benchmarks (merge-sort), as well as trivial programs, and entirely simulated computational graphs (which do not actually compute anything, just simulate a computation)

We look at the expected results from the model described above, and then we examine the output of the simulator.
We should see a strong correlation between the simulated results and the expected results.


\bibliographystyle{plain}
\bibliography{biblio}

\end{document}
