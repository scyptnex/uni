%%%%%%%%%%%%%%%%%%%%%%%%%%%%%%%%%%%%%%
% A presentation by Gilles Bertrand  %
% gilles.bertrand at ieee dot org   %
% December 2006            %
%%%%%%%%%%%%%%%%%%%%%%%%%%%%%%%%%%%%%%

\documentclass[pdf]{beamer}
\usetheme{Berkeley}

\usepackage{time}       % date and time
\usepackage{graphicx}
\usepackage[T1]{fontenc}    % european characters
%\usepackage{courier}
\usepackage{amssymb,amsmath}  % use mathematical symbols
\usepackage{palatino}                  % use palatino as the default font
\setbeamercovered{transparent}

\begin{document}

% here you define the information that will be displayed in the title/cover page
\title[QoS management in IMS]{QoS management in the IP Multimedia Subsystem\\}
\subtitle {Architecture, protocols and QoS management in IMS}
\author[Gilles Bertrand]{Gilles Bertrand\\TELECOM Bretagne - RSM Department\\
\vspace*{0.5cm}    
\includegraphics[height=0.5cm]{./figures/e-mail}
}
\date{11th December 2006}

% this is used in the pdf information
\subject{QoS management in the IP Multimedia Subsystem (IMS)}

% here you build the title page
\frame{
 \titlepage
}

% outline
\AtBeginSection[]
{
 \begin{frame}
  \frametitle{Outline}
  \small
  \tableofcontents[currentsection,hideothersubsections]
  \normalsize
 \end{frame}
}



\section*{Introduction}
\begin{frame}
    \frametitle{Trends in the Internet}
    \begin{itemize}
        \item The IP is ubiquitous
        \item Services with high QoS requirements   gain momentum
        \item Value lies in services
        \pause
        \item[$\Rightarrow$] Currently deployed networks need to adapt to these tendencies.
     \item[$\Rightarrow$] The IP multimedia Subsystem is seen as a promising solution for fulfilling these needs.
    \end{itemize}
\end{frame}

%\part{Main Talk} % used for two reasons: structuring and decreasing outline granularity
\begin{frame}
    \frametitle{The IMS standards}
     \begin{block}{3GPP}
      initiated at the work on IMS (Release 5 - March 2003), focused on facilitated service development and deployment.
      3GPP R7 underway.   
     \end{block}
     \pause
     \begin{block}{ETSI TISPAN}
        Extended IMS focus for network convergence purposes (Access agnostic) in the scope of the work on Next Generation Networks (NGN).
        First release available. Second release underway.
     \end{block}
     \pause
     \begin{block}{IETF}
        In order to facilitate interoperability IMS specifies the use of open Internet protocols standardized at IETF.
     \end{block}
\end{frame}


\section{Motivation for the use of IMS}
    \subsection{Basic principles}
    \begin{frame}
        \frametitle{Easier network management}
        \begin{itemize}
            \item Separated control and bearer functions
            \item All IP overlay service delivery on top of a packet switched architecture
            \item Migration of Circuit Switched services to Packet Switched domain
            \pause
            \item[$\Rightarrow$] Network management savings
        \end{itemize}
    \end{frame} 
   
% \textsubrightarrow 
   
    \begin{frame}
        \frametitle{End-to-end architecture}
        \begin{itemize}
            \item Service delivery independent of the access technology
            \item Use of open Internet protocols
            \item End-to-end QoS management (support for real-time services)
            \pause
            \item[$\Rightarrow $] Inherent contradiction: IMS relies on IP technologies allowing free communications but aims at controlling IP service delivery.
        \end{itemize}
    \end{frame}

%\begin{itemize}[<+->]  => piecewise undercover
   
    \begin{frame}
        \frametitle{Horizontal architecture}
        \begin{overprint}
            \onslide<2->{
                \begin{itemize}       
                    \item Horizontal service integration
                    \item Use of service enablers for several services
                \end{itemize}
            }
            \onslide<1->{
                \begin{center}
                    \begin{figure}
           \scalebox{0.50}{
            \includegraphics{./figures/horizontal_vs_vertical_services_integration}
           }
           \caption{Horizontal vs vertical service integration}
           \end{figure}
                \end{center}
        }
        \onslide<3->{
            \begin{itemize}
                    \item[$\Rightarrow$] A thigh interaction of services is possible
                    \item[$\Rightarrow$] Easier and faster service development
                \end{itemize}
            }        
        \end{overprint}
    \end{frame}   
   
    \subsection{Business motivation}
    \begin{frame}
        \frametitle{IMS: an answer to operators concerns}
     \begin{block}{Operators concerns and expectations}<1->           
            \begin{itemize}
                \item The Average Revenue Per User decreases
                \item Internet actors take their benefits from the network infrastructure
                \item[$\Rightarrow$] Operators want to be more than Bitpipes
            \end{itemize}
        \end{block}
     \begin{block}{IMS features}<2->               
            \begin{itemize}
                \item Central role of the operators in service delivery
                \item Network infrastructure and management savings
                \item Cost effectiveness
            \end{itemize}
        \end{block}
    \end{frame}
   
    \begin{frame}
        \frametitle{IMS: an answer to operators concerns}
     \begin{block}{Additional expectations of the operators}<1->
            \begin{itemize} 
                \item Broaden the commercial offer (Apply suitable billing schemes)
                \item Decrease the service time-to-market
                \item Offer innovative and attractive services
            \end{itemize} 
        \end{block}
                       
     \begin{block}{Additional benefits of IMS}<2->           
            \begin{itemize}
                \item Fine knowledge of the services used by a customer
                \item Easier and faster development/deployment of multimedia services
                \item Smaller investment threshold for new service deployment             
                \item Interaction of services
                \item Several synchronized multimedia services in a single session

            \end{itemize} 
        \end{block}
    \end{frame}   
   \subsection{Business issues}
    \begin{frame}
        \frametitle{Business Issues related to IMS}
        \begin{itemize}
            \item<+-> New business model (Operators play a central role is service delivery)
            \item<+->[$\Rightarrow$] Competition with Internet world actors
            \item<+-> Multimedia service delivery by network operators
            \item<+->[$\Rightarrow$] Operators have to obtain content
            \item<+-> When should IMS be deployed? (early $\rightarrow$ high margins but risky, late  $\rightarrow$ established standards but high competition).
        \end{itemize}         
    \end{frame}   
       
   \subsection{Technical issues}
        \begin{frame}
        \frametitle{technical Issues} 
            \begin{itemize}
                \item QoS $\rightarrow$ How to provide end-to-end QoS over heterogeneous networks?
                \item Privacy $\rightarrow$ How should third parties access user profiles?\ldots   
                \item Interoperability $\rightarrow$ how to manage different access constraints,
 terminal capabilities\ldots
                \item Several important subsystems are not part of the IMS (P2P, VPNs, SMS, IPTV\ldots).
            \end{itemize}
        \end{frame}     



%%%%%%%%%%%%%%%%%%%%%%%%%%%%%%%%%%
% Presentation de l'architecture %
%%%%%%%%%%%%%%%%%%%%%%%%%%%%%%%%%%

\section{IMS Architecture}
    \subsection{Overview}
    \begin{frame}
        \frametitle{IMS layered model}
        \begin{overprint}
            \onslide<2->{
                \alert{IMS is a \emph{functional} architecture}
                \begin{itemize}
                    \item Application layer (HSS, AS)
                    \item Control layer (IMS core, PSTN emulation, IPTV,\ldots)
                    \item Transport layer (UE, access network, NGN core, NASS, RACS, \ldots)
                \end{itemize}
            }
            \onslide<1->{         
                \begin{center}
                    \begin{figure}
                \scalebox{0.75}{
               \includegraphics{./figures/archi_IMS_collapsed.png}
           }
           \caption{IMS Architecture, overview}
           \end{figure}     
                \end{center}
            }   
        \end{overprint} 
    \end{frame}
   
    \subsection{The IMS core}
    \begin{frame} 
        \frametitle{The IMS core (TISPAN)}
     \begin{overprint}     
         \begin{block}{Function}<1->
            Session and media control
         \end{block}                             
         \begin{block}{Components}
            \begin{itemize}
                    \item Call Session Control Function (CSCF)
                    \item Multimedia Resource Function Controller (MRFC) and Processor (MRFP)
                    \item Breakout Gateway Control Function (BGCF)
                    \item Media Gateway Control Function (MGCF)             
                \end{itemize}
         \end{block}
        \end{overprint}
    \end{frame}       
    \begin{frame} 
        \frametitle{The IMS core: CSCFs}
     \begin{overprint}     
        \only<1->{
             \begin{block}{CSCFs}
                \begin{itemize}
                        \item Serving CSCF (S-CSCF): controls the communication session, invokes the AS. Located in home network.
                        \item Proxy CSCF (P-CSCF): IMS contact point for SIP user agents. May include a Policy Decision Function (PDF).
                        \item Interrogating CSCF (I-CSCF): gateway to other domains. Used for Topology Hiding (THIG) or if several S-CSCF are located in the domain.
                    \end{itemize}
             \end{block}                             
            }   
     \end{overprint}
    \end{frame}     
    \begin{frame} 
        \frametitle{The IMS core: MRFC, MRFP, MGCF}
     \begin{overprint}     
        \only<1->{
             \begin{block}{MRFC}
                    Controls the MRFP
             \end{block}                             
            }           
         \begin{block}{MRFP}
                Provides transcoding and content adaptation functionalities.
         \end{block}
         \begin{block}{MGCF}
                Controls a media gateway.
         \end{block}               
     \end{overprint}
    \end{frame}     

   
    \begin{frame}
        \frametitle{The IMS core: functional internal architecture}   
        \begin{center}
            \begin{figure}
        \scalebox{0.75}{
        \includegraphics{./figures/IMS_core.png}
    }
    \caption{IMS core}
    \end{figure}     
        \end{center}
    \end{frame}     

   
    \subsection[NASS \& RACS]{The NASS and RACS}
    \begin{frame}
        \frametitle{NASS internals}
     \begin{block}{Function}       
        Provision of IP addresses and network parameters dynamically. Play the role of a DHCP server, a Radius client and provides location management.
     \end{block}         
        \begin{center}
            \begin{figure}
            \scalebox{0.75}{
           \includegraphics{./figures/NASS_internal.png}
       }
       \caption{NASS internal  }
        \end{figure}
        \end{center}         
    \end{frame}     
   
    \begin{frame}
%        \frametitle[NASS subfunctions]
     \begin{block}{Network Attachment SubSystem (NASS) subfunctions}       
            \begin{itemize}
                \item<+-> Network Access Configuration Function (NACF)
                \item<+-> Access Management Function (AMF)
                \item<+-> Connectivity Session Location and Repository Function (CLF)
                \item<+-> User Access Authorization Function (UAAF)
                \item<+-> Profile Data Base Function (PDBF)
                \item<+-> Customer Network Gateway (CNG) Configuration Function (CNGCF).
            \end{itemize}
     \end{block}
     
     \begin{overprint}
        \onslide<1>
             \begin{block}{NACF}
                    Responsible for IP address allocation to the UE           
                    May provide additional parameter
                \end{block}
        \onslide<2>
             \begin{block}{AMF}
                Interface equipment between Access Network and NACF
                \end{block}             
        \onslide<3>
             \begin{block}{CLF}
                Associates IP address to user location information. It may store further information.
                \end{block}     
        \onslide<4>
             \begin{block}{UAAF}
                Performs authentication for network access based on the user profile stored in the PDBF
                \end{block}     
        \onslide<5>
             \begin{block}{PDBF}
             Stores the user profile and authentication data.   
                \end{block}     
        \onslide<6>
             \begin{block}{CNGCF}
                Controls the Customer Network Gateway when necessary
                \end{block}                                                                                     
     \end{overprint}
    \end{frame}     


\section*{Bibliography}
    \bibliography{./biblio_gilles_bertrand}
\end{document}
